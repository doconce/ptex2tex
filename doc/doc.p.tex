%\documentclass[envcountsect]{lncse}
\documentclass[a4paper,11pt]{article}
\usepackage{ptex2tex}
\usepackage{pythonhighlight}
\begin{document}
% TITLE: ptex2tex tutorial\n
% AUTHOR: Ilmar M. Wilbers, Simula Research Laboratory, Univ. of Oslo\n
% DATE: Mar 25, 2008\n

\title{Ptex2tex: Flexible Handling of Computer Code in \LaTeX{} Documents}
\author{Ilmar Wilbers \and Hans Petter Langtangen}

\date{Simula Research Laboratory\\ 
Email: \texttt{\{ilmarw,hpl\}@simula.no}\\
\ \ \\
\today}
\maketitle

\section{Introduction}
\subsection{What is Ptex2tex?}
Ptex2tex is a tool that allows you to replace \LaTeX~environment declarations
with simple keywords. Say that you are writing a book or an article, and there
is a certain style of box that you use frequently. The annoying thing is that
this box has a lot of \LaTeX~syntax connected with it in order for it to look like
the way you want it to. Let's look at an example involving a blue box:
\bsni
import sys; print 'script name is', sys.argv[0]
\esni
The \LaTeX~source code for this box looks as follows:
\begin{verbatim}
\providecommand{\shadedquoteBluesnippet}{}
\renewenvironment{shadedquoteBluesnippet}[1][]{
\definecolor{shadecolor}{rgb}{0.87843, 0.95686, 1.0}
\definecolor{shadetitle}{rgb}{0.5, 0.95686, 1}
\bgroup\rmfamily
\fboxsep=0mm\relax
\begin{shaded}
{{\hfill\tiny\textsf{\textcolor{shadetitle}{Snippet\ \ }}}}
\list{}{\parsep=-2mm\parskip=0mm\topsep=0pt\leftmargin=2mm
\rightmargin=2\leftmargin\leftmargin=4pt\relax}
\item\relax}
{\endlist{\textcolor{shadecolor}{\ }}\end{shaded}\egroup}
\begin{shadedquoteBluesnippet}
\fontsize{9pt}{9pt}
\begin{Verbatim}
import sys; print 'script name is', sys.argv[0]
\end{Verbatim}
\end{shadedquoteBluesnippet}
\noindent
\end{verbatim}
\noindent
With Ptex2tex, the box is crated by just enclosing the text (inside the box)
in a certain \emph{environment}, say \code{mybox}:
\bdat
\bmybox
import sys; print 'script name is', sys.argv[0]
\emybox
\edat
The \LaTeX{} code corresponding to the \code{mybox} environment, i.e.,
how the box should look like, can now be defined in a configuration file. 
It is then easy to work with boxes, typically contining computer code,
in a large document and with Ptex2tex just use some environment names
rather than the full, complicated \LaTeX{} surroundings.
That is,
the
Ptex2tex approach allows much less code, and makes it easier to concentrate on the contents
of the document we are writing and not the \LaTeX~code necessary to create such
a box. Suppose you want the box to have a completely different look:
\bcod
import sys; print 'script name is', sys.argv[0]
\ecod
which in pure \LaTeX~looks like:
\begin{verbatim}
\providecommand{\shadedskip}{}
\definecolor{shadecolor}{rgb}{0.87843, 0.95686, 1.0}
\renewenvironment{shadedskip}{
\def\FrameCommand{\colorbox{shadecolor}}\FrameRule0.6pt
\MakeFramed {\FrameRestore}\vskip3mm}{\vskip0mm\endMakeFramed}
\providecommand{\shadedquoteBlue}{}
\renewenvironment{shadedquoteBlue}[1][]{
\bgroup\rmfamily
\fboxsep=0mm\relax
\begin{shadedskip}
\list{}{\parsep=-2mm\parskip=0mm\topsep=0pt\leftmargin=2mm
\rightmargin=2\leftmargin\leftmargin=4pt\relax}
\item\relax}
{\endlist\end{shadedskip}\egroup}\begin{shadedquoteBlue}
\fontsize{9pt}{9pt}
\begin{Verbatim}
import sys; print 'script name is', sys.argv[0]
\end{Verbatim}
\end{shadedquoteBlue}
\end{verbatim}
\noindent
Using plain \LaTeX, you would have to replace the \LaTeX~code, which in this
case is located both before and after the code you want to display. In
addition, you may have several hundred boxes throughout your document. Changing
all the \LaTeX~code for these boxes would require a lot of work, even if you
use search and replace in your editor. Using Ptex2tex you could
simply swap \code{mybox} with \code{yourbox}, or change the configuration file
accordingly if the change is to be made for all boxes using this style.
In this way, you do not need to
make any changes to the document itself at all!

Another major advantage of Ptex2tex is that it provides means to include text from files as well as
output from programs ran at the commandline at compile time. Say
that you are writing a book about programming or a manual for a programming
tool. You frequently include computer program
code in your document, and also the output of running programs. While writing
the document, you might make changes to programs and the output of
them. Instead of cutting and pasting both the actual program and the output
every time it is changed, using Ptex2tex, the only thing you need to do
is to run \code{ptex2tex} on the document. Let's look at an example. We
include the following in the \code{.p.tex} document:
\bdat
 @@@CODE division.py def@if
 The output from running this function with a=2 and b=0 is:
 @@@CMD python division.py #0
\edat
Now, the file \code{division.py} itself as well as the output from running it
will be included in our document, and the result looks as follows:\\
\line(1,0){374}\\
@@@CODE division.py def@if
The output from running this function with a=2 and b=0 is:
@@@CMD python division.py #0
\line(1,0){374}\\
 \\
This makes Ptex2tex a very convenient tool, particularly 
when writing large \LaTeX~documents about computer programming where
one needs to include a lot of programs and code snippets.

\subsection{Motivation}
Why do we not simply define \LaTeX~environments and use these directly in our
files? We consider Python more powerful and convenient than \LaTeX~for
generating environments. Also, including text from file using search
expressions as well as
including results from programs ran at the commandline are very challenging, if
not impossible, tasks in \LaTeX. Imagine that you want to include a certain
paragraph or code block from a file. Using plain \LaTeX~this means copying the
necessary text from the source file, and pasting it into the \LaTeX~file. If
you make changes to the file you copied from, you have to remember to copy the
text you need a second time. Even though there exist ways of including whole
files in \LaTeX~(\code{verbatiminput}), it is very difficult to include a part
of the file. Also, you might want the text to be included to
be typeset in a certain way, something that cannot be achieved by a simple
\code{verbatiminput}. The point is that even though these things \emph{could}
be done in \LaTeX~using classes, for instance, writing them in Python is much
easier. As Ptex2tex uses Python configuration files to define commands, one does not even have to
know Python in order to extend and tailor Ptex2tex.

\subsection{Structure of this tutorial}
This tutorial is divided in four sections. Following the introduction is a
section about preprocessor statements, followed by a section on including files
and output from program execution. Finally, we will explain how the
Ptex2tex environments work and how we can configure them.

\section{The prepocessing step}
It is important that the extension of the files we are working with is
\code{.p.tex}. In this way, all (most) editors recognize the files as normal
\LaTeX~files, with all advantages implied.
We will start by looking at what happens when we run the command
\code{ptex2tex test.p.tex} on the commandline. If the extension of
the file is not \code{.p.tex}, the program will exit with an error
message. If no extension is given, that is, we run \code{ptex2tex test}, it
is assumed that we are looking for the file \code{test.p.tex}.

% #define INCLUDE_PREPROCESS 1
% #ifdef INCLUDE_PREPROCESS
The first step of Ptex2tex consists of running Trent Mick's \code{preprocessor}. This is a Python
package that allows us to use preprocessor statements in files, just like the
preprocessor statements that are known from the \code{C} language. These
statements take the form of normal comments in the source file, but when
running \code{preprocessor}, these are treated in a special way. For
instance, we can include whole files, or include certain blocks of
test or code in our file only when a special requirement is met, otherwise this text
or code is ignored. The statements need to be on a
separate line. The statements that are implemented in the current version
(which is 1.1.0 as of January 2009), are:
\begin{itemize}
\item
\code{% #define VAR [VALUE]}
\item
\code{% #undef VAR}
\item
\code{% #ifdef VAR}
\item
\code{% #ifndef VAR}
\item
\code{% #if EXPRESSION}
\item
\code{% #elif EXPRESSION}
\item
\code{% #else}
\item
\code{% #endif}
\item
\code{% #error ERROR_STRING}
\item
\code{% #include "FILE"}
\end{itemize}
The source code for the present document, \code{doc.p.tex}, includes a few of these
statements as an example. (For now, it is not possible to include arguments to
\code{preprocess} within Ptex2tex, meaning that only the input file and output
file are specified, with an additional argument to force the preprocessing
 by using the \code{-f} flag for \code{preprocess}). For further documentation, please
visit the website for this
package\footnote{\texttt{http://trentm.com/projects/preprocess}}. The input file
type to \code{preprocess} is \code{.p.tex} and the output is \code{.tmp1}.  If
\code{preprocess} is not available on the system, this step is skipped. Preprocessor
statements in the file will then be treated as comments, and the file is
copied directly to \code{.tmp1}.
% #else
Information about the preprocess stage is left out.
% #endif

\section{Including files and output from program execution}
The second step is to examine and execute the Ptex2tex keywords for
including files and for including the output from executed programs. For both
these two stages, the input file type will be \code{.tmp1} and the output file
type is \code{.tmp2}.

\subsection{\texttt{@@@CODE}}
A line \emph{starting} with this statement indicates that a file containing a program
(that is, computer code) is to be included in the document. If there is only
one argument after the statement (the file name), the whole file is included. If a second
argument is included, it is split with respect to the character
\code{'@'}. If only an expression in front of this character is given, 
only the part of the file starting with that
argument and ending with the end of the file is included. If an expression
after the character is given as well, the part of the file starting with the
first expression and ending at the beginning of the second expression is
included. This means that the included part ends \emph{before} the text in the second
expression, hence that text is \emph{not} included. When searching the file, the
first occurrence of the start expression is used. Ptex2tex then starts scanning
for the stop expression from the beginning of the location of the start expression,
which will denote the end of the
region to be copied into the text. White-spaces in front of and at the end of the
expressions before and after \code{'@'} are ignored.

The searching in the file for start and stop expressions is done using the
Python function \code{find} on strings. Hence, start/stop expressions are
compared directly to the text in the file. 
% #define INCLUDE_CTEX 0
% #ifdef INCLUDE_CTEX
In the \code{ctex2tex} script,
which Ptex2tex is based on, regular expressions were used.
% #endif

The environment keys 'pro'
and 'sni' are short for 'program' and 'snippet', respectively, indicating that
the former is meant to be used for typesetting a complete program, whereas the
latter is meant for a program snippet, for instance a function. Both these
Ptex2tex environments are for typesetting code. Environment keys
here means keywords that Ptex2tex substitutes with actual \LaTeX~code
later. More information about Ptex2tex environments is given in section \ref{sec:config}.
If the whole file is included, the Ptex2tex environment key 'pro' is used. Otherwise,
the environment key 'sni' is used. An exception here is in the case that a
second \code{'@'} is used after the stop expression. This indicates that the 'pro'
environment key is to be used, instead of 'sni'. The reason one might wish to
typeset only a part of a file as a full program, is that one can use the
search expressions for removing details in the full program, such as headers
and test functions, that are irrelevant or confusing in the document, but
still are considered important for the actual program.

It is important to differ between the \code{@@@CODE} statement and the include
statement provided with the \code{preprocessor} package. The latter allows us to
include a file by \code{% #include FILE}, but does not embed the copied text
in special \LaTeX~environments, that is, the whole file is included as is.

We will exemplify this for the file \code{myprog.py}. 
Let's look at the output of some different options:\\
\code{'@@@CODE myprog.py'}
includes the whole file:
@@@CODE myprog.py
\code{'@@@CODE myprog.py def myfunc'} and \code{'@@@CODE myprog.py def myfunc @'}
includes everything from the first instance of the string 'def myfunc'
to the end of the file:
@@@CODE myprog.py def myfunc @
\code{'@@@CODE myprog.py if isinstance@elif'}
includes everything from the first instance of the string 'if isinstance'
\emph{up to} , but not including, the first instance of 'elif' \emph{after} 'if isinstance'.
@@@CODE myprog.py if isinstance@elif

\subsection{\texttt{@@@DATA}}
\code{@@@DATA} is essentially the same as \code{@@@CODE}, but different environment keys are used:
'pro' becomes 'dat' and 'sni' becomes 'dsni'. These are short for 'Data' and
'Data snippet', respectively. Whereas \code{@@@CODE}
is meant to be used with program code, \code{@@@DATA} is meant to be used with
data files that are not code, for instance input files to, and output files
from, computer programs.

\subsection{\texttt{@@@CMD}}
A line \emph{starting} with the statement \code{@@@CMD} indicates that we want to include
the output from a shell command with commandline arguments. All text after
the statement \code{@@@CMD} will be executed, except the text after (and including)
\code{'#'}. The character after \code{'#'} defines how much of the commands we
want to execute should be part of the final text to be included in our
document. Any whitespaces after the character are ignored. There are five possible values:
\begin{enumerate}
\item
\code{'0'} omits the whole command that is to be executed
\item
\code{'1'} the command is included and the full path stripped
\item
\code{'2'} the command is included and the full path is not stripped
\item
\code{'3'} the command is included except for the name of the program being
called (the first word) and the full path is stripped
\item
\code{'4'} the command is included except for the name of the program being
called (the first word) and the full path is not stripped. 
\end{enumerate}
For options \code{'1'} and \code{'3'} a simple regular expression search is used
to allow statements like \code{'python code/myprog.py'}
to be printed as \code{'python myprog.py'} and \code{'myprog.py'},
respectively, that is, we remove the path to the directory where
the program is located, allowing us to run programs in a different directory
than where we are running Ptex2tex, without the output reflecting this.
For options \code{'3'} and \code{'4'} the first word is simply stripped,
allowing statements like \code{'python code/myprog.py'} to be printed as \code{'myprog.py'}
and \code{'code/myprog.py'}, respectively. The default option is \code{'3'},
and the terminal session is typeset using the environment key 'sys' (for ``system'').

Again, let us exemplify this. Running \code{'python myprog.py -1 2 -5'} in a
shell gives us the following output:
@@@CMD python myprog.py -1 2 -5 #0
Instead of copying and pasting this into the documents we are creating,
meaning we would have to update this every time the code in \code{myprog.py}
is changed, we simply add the following line:
\code{'@@@CMD python myprog.py -1 2 -5'}. The result looks like this:
@@@CMD python myprog.py -1 2 -5
Before looking at the additional options, let us assume that the file
\code{myprog.py} is located in a subfolder \code{myfolder}, meaning that
we would have to run the command \code{'python myfolder/myprog.py -1 2 -5'}.
Adding additional options to this commands gives the following results:\\
\code{'@@@CMD python myfolder/myprog.py -1 2 -5 # 0'} omits the whole command
that is called:
@@@CMD python myfolder/myprog.py -1 2 -5 # 0
\code{'@@@CMD python myfolder/myprog.py -1 2 -5 # 1'} removes the additional folders:
@@@CMD python myfolder/myprog.py -1 2 -5 # 1
\code{'@@@CMD python myfolder/myprog.py -1 2 -5 # 2'} doesn't remove anything:
@@@CMD python myfolder/myprog.py -1 2 -5 # 2
\code{'@@@CMD python myfolder/myprog.py -1 2 -5 # 3'} removes the additional
folders and the first word after \code{'@@@CMD'} (which is \code{'python'}):
@@@CMD python myfolder/myprog.py -1 2 -5 # 3
\code{'@@@CMD python myfolder/myprog.py -1 2 -5 # 4'} removes only the first
word after \code{'@@@CMD'}:
@@@CMD python myfolder/myprog.py -1 2 -5 # 4

We give a table with an overview of the different Ptex2tex environments
that the \code{@@@} keywords are mapped to in Table \ref{table:env}.

\begin{table}
\begin{center}
\begin{tabular}[h]{|l|r|}
  \hline
  Keyword & Ptex2tex key\\
  \hline
  @@@CODE without search expressions & pro\\
  @@@CODE with \code{start} or \code{start@} & sni\\
  @@@CODE with \code{start@stop} & sni\\
  @@@CODE with \code{start@@} & pro\\
  @@@CODE with \code{start@stop@} & pro\\
  @@@DATA without search expressions & dat\\
  @@@DATA with \code{start} or \code{start@} & dsni\\
  @@@DATA with \code{start@stop} & dsni\\
  @@@DATA with \code{start@@} & dat\\
  @@@DATA with \code{start@stop@} & dat\\
  @@@CMD  with any parameters & sys\\
  \hline
\end{tabular}\caption{Mapping of keywords to environments}\label{table:env}
\end{center}
\end{table}

\section{Ptex2tex environments}\label{sec:config}
This section explains the use of the Ptex2tex
environments. Ptex2tex environments consist of text that we want to
typeset in a special way (typical computer code), surrounded by keywords for
defining the beginning and the end of the text to be
typeset. Ptex2tex replaces these keywords with the \LaTeX~commands
that we have configured the Ptex2tex environment in question to be
associated with. We will now explain this in more detail.

\subsection{How to use the environments}
Ptex2tex environments are described by several keywords. In the
previous section we looked at the ones starting with \code{'@@@'}, and
explained that these keywords were replaced with different Ptex2tex
environment keys. These environment keys can have different names, we already
encountered 'pro', 'sni', 'dat', and 'sys'. These are environment keys that are
built into Ptex2tex and need to be there, as they are used for the
\code{'@@@'} keywords, but we can define what the \LaTeX~environment they are
translated to should look like, as we
will see shortly. In addition, we can add any new environment keys as we
please, and some are already defined by default.

Let us have a closer look at the environment key 'pro'. For every environment
key, two keywords can be used in the document we are working with. For 'pro'
these would be \code{\}\code{bpro} to indicate where we want to start the
typesetting for that environment, and the other is \code{\}\code{epro},
indicating the end. In general, for an environment key 'xxx' there will be
keywords \code{\}\code{bxxx} and \code{\}\code{exxx} where 'b' and 'e' stand
for 'begin' and 'end', respectively. When Ptex2tex is scanning the
documents, these two keywords are replaced with the actual \LaTeX~code that
makes up the environment. Let us once again look at an example. The lines
\bdat
 \bpro
 //This text is typeset as computer code.
 \epro
\edat
will result in the following box:
\bpro
//This text is typeset as computer code.
\epro
The advantage is that we now can change the way this box should look like
without making any changes in the document itself, as described in the next
section. Also, if we want to change the environment to be used, say for
instance from 'pro' to 'sys', instead of changing up to several lines of
\LaTeX~code, we simply change the keywords from  \code{\}\code{bpro} and
\code{\}\code{epro} to \code{\}\code{bsys} and \code{\}\code{esys}.
This step results in a \code{.tex} file that can be compiled with \LaTeX. In
the next section we show more of the details behind the environment keys.

\subsection{The configuration file}\label{sec:conffile}
As mentioned earlier, the idea of Ptex2tex is to allow the user to write short keywords for
parts of the document that are to be typeset in a special way. When running
Ptex2tex, these keywords are then replaced with the full
\LaTeX~commands, resulting in plain \code{.tex} files. For instance may the
keyword \code{\}\code{bpy} be replaced with
\bdat
\plin
\edat
and
\code{\}\code{epy} with
\bdat
\elin
\edat
Some keywords will also output additional \LaTeX~code for defining
\LaTeX~environments once for every file, more on this later. We will differentiate
between Ptex2tex environments, which we are about to explain in detail,
and \LaTeX~environments, which refer to the \LaTeX~commands that we would
have to use instead of the Ptex2tex keywords.

All Ptex2tex environments available are defined in a configuration
file. When running Ptex2tex for the first
time, or if the file is removed since the last time Ptex2tex was run,
this configuration file will be placed in the user's home directory. The file
is named \code{.ptex2tex.cfg}, indicating that it is hidden. The file is
written in the Python ConfigParser
language\footnote{http://docs.python.org/lib/module-ConfigParser.html} and
will be referred to as the global environment file.
This file is read every time one runs Ptex2tex. In addition to
this file, the \code{.ptex2tex.cfg} file in the directory where the \code{.p.tex}
file currently being processed is located, is also evaluated, if it exists. This file is
referred to as the local environment file. The global environment
file is evaluated first, and then the local one. This means that if a
specific environment is defined both in the global and local environment
files, the environment from the local file will overwrite the one from the
global file.

In this way, one can make changes to an environment that will only affect the
local \code{.p.tex} files. At the same time, one can make changes to the
global file, but these changes will only matter if the environment it concerns
is not overwritten in the local environment files. One can thus add
environments to the global file and they will be available when running Ptex2tex
from any location. 

The configuration file is made up of different sections, one for each
environment and one for the mapping of environment keys to the
corresponding environment. Each environment can contain the following options:
\bdat
breplace
ereplace
newenv
define
fontsize
bstretch
\edat
We refer to the \emp{ptex2tex.cfg} file in the \emp{lib/ptex2tex}
directory to see a large number of examples of how these environments
can be used.

Each section has a heading enclosed by brackets. This heading will also be the
name of the environment. The options for this environment should follow on the
lines below. For instance, to define the Ptex2tex environment 'Extra'
and the option \code{breplace}, we would write:
\bplin
[Extra]
breplace = \begin{Verbatim}
\eplin

In addition to the sections defining the environments, there is a section
\code{[names]}. In this section we map environment keys to
Ptex2tex environments. For instance, if we want the key 'pro' to be associated with
the environment 'Bluebox', we write:
\bdat
[names]
pro = Bluebox
\edat
Hence we can use \code{\}\code{bpro} to indicate the
beginning of the 'pro' environment, and \code{\}\code{epro} to indicate the end.
We can set multiple keys to point to the same environment:
\bdat
[names]
pro = Bluebox
tmp = Bluebox
\edat
If we try to assign multiple environment to the same key, only the last one is
used:
\bdat
[names]
pro = Bluebox
pro = Graybox
\edat

\subsection{Defining new environments}
A Ptex2tex environment in the configuration file is defined by a
section heading and several options. A minimum requirement of an environment is
that the options \code{breplace} and \code{ereplace} are defined. Hence, for
the environment 'Extra', the following lines should exist and be defined:
\bplin
[Extra]
breplace = \begin{verbatim} 
erplace = \end{verbatim}
\eplin
In this case they define a simple \code{Verbatim} environment. \code{breplace}
and \code{ereplace} control what \LaTeX~commands that should be returned at the
beginning and at the end of the Ptex2tex environment, respectively.
Note that only defining the environment is not enough, the keywords to be used
with this environment also need to be defined in the 'names' section:
\bdat
[names]
ext = Extra
\edat
Now, the keywords \code{\}\code{bext} and \code{\}\code{eext} will
be replaced with the text defined in \code{breplace} and \code{ereplace}.

We differ between two categories of environments in Ptex2tex: 
those that make use of the \code{\}\code{newenvironment} keywords in \LaTeX,
and those who don't, but instead use other \LaTeX~
commands like \code{\}\code{minipage} and different packages like
\code{fancyverb} or \code{framed} as well as different box environments. 

The problem with Ptex2tex environments that use \code{\}\code{newenvironment} is
that \LaTeX~needs this environment to be defined, but not more than
once, or it will issue an error. Therefore, when replacing the first occurrence of each environment
keyword (for instance 'bpro'), a definition is added as well. This is the
keyword \code{newenv} in the section for an environment. Default, this
is an empty string. When running Ptex2tex
on multiple files, and then combining these files in a single document, this
environment will be defined multiple times, once for every \code{.p.tex}
document. Since we run Ptex2tex on one file at the time, Ptex2tex cannot know
if a specific environment is defined earlier. 

In order to get around this
limitation, we use \code{\}\code{renewenvironment} instead of
\code{\}\code{newenvironment}. But since it is not possible to renew an
environment unless it already is defined, this by itself does not solve the
problem. But if we first add the line
\bdat
\providecommand{\shadedquote}{}
\edat
we have created a way around this. What it does is to create a command
\code{shadedquote} given that \code{shadedquote} is not already defined. There
is no equivalent\\ (\code{\}\code{provideenvironment}) for environments, but
since \code{\}\code{renewenvironment} simply checks if the name of the
environment (\code{shadedquote}) is defined, not if it is defined as an
environment, this works. 

When defining new environments as described above, the name of the
environment must be unique in the configuration file. For example,
``shadedquote'' must not be defined in another environment.
Ptex2tex is normally able to detect such problems.

It is also possible to use an environment that is
defined elsewhere, say in third-party \LaTeX~package. One can then use the option
\code{define}. If it is set to \code{False}, it is assumed that the
\LaTeX~environment is defined externally, and Ptex2tex will not define
this environment. 

There are two types of boxes supported by the standard configuration
file \emp{ptex2tex.cfg}. One kind (e.g., 'Blue') allows the text 
in the box to be split over pages, while the other kind (e.g., 'Blue\_sp')
does not allow splitting, which often forces \LaTeX\ to move the box
to the next space, leaving a lot of white space on the preceding page.
% outdated:
%Currently the
%Ptex2tex environments 'Program' and 'Data' use
%\code{\}\code{renewenvironment}. The names of the \LaTeX~environments are
%\code{shadedquote} and \code{shadedquotedat}, respectively.
%These two environments are also the only ones that support code spanning multiple
%pages. This means that if the text between the beginning and ending keywords
%is too long to fit on a single page, it is divided over several pages. With the
%other Ptex2tex environments, the box enclosing the text might be
%moved to the next page, leaving a lot of white space. In the case that the
%text within the box exceeds one page, it will even disappear below the
%bottom of the page. This is a known problem for those who have experience in
%using \LaTeX. 

Please note that the names of the environments should not
contain numbers, only letters. Naming an environment
\code{\}\code{shadedquote2} makes the interpreter think that we are using
\code{\}\code{shadedquote} followed by the number $2$.

It is possible to use so-called variable interpolation in the configuration
file. This means that one can use the name of one option in another option in
the section, and the value for that option will then be included in the first
option, for example:
\bdat
[SomeBox]
breplace = \begin{Verb}[baselinestretch=%(bstretch)s]
bstretch=0.85
\edat
What happens here is that the \code{%(bstretch)s} part is substituted with the value
of the option 'bstretch'. The \code{'\%'} indicates that the variable within the
following parenthesis is to be interpolated. The 's' after the last
parenthesis means that the option to be inserted is a string. See the
\emp{ptex2tex.cfg} file in \emp{lib/ptex2tex} for numerous examples.

Often, the option text for 'breplace' or 'ereplace' spans several lines (of
\LaTeX~code). It is then important that the text for the lines following the
first use the same indentation. It is possible to comment out some of the
lines in the configuration file by using a \code{'#'} at the beginning of the
line. Any other position won't work, as it will be parsed as part of the
option instead. If you need to use comments in the \LaTeX~code, you need to
use two \code{'\%'}, as only one will be considered as variable interpolation,
and will likely cause an error. In the previous code example
we show some of the mentioned points.

For further documentation of Python Config Files, see the Python Library
Reference.

All options within a section of the configuration file are parsed within
Ptex2tex. You can define your own options beyond the six mentioned,
but this will result in a warning.
%If you try to run \code{ptex2tex -h}, you will see that for every environment
%that is found, there are two command line arguments that are available. The
%names of these are the keyword for the class followed by a \code{f} and a
%\code{m}. For \code{pro}, these are thus \code{--prof} and
%\code{--prom}. These stand for 'fontsize' and 'margin', respectively. Although
%they are not currently used with any of the default environments that are
%included with Ptex2tex, they are available as the class attributes
%\code{fontsize} and \code{margin}. If they are not set at runtime, their
%default value is \code{None}.

\subsection{Remarks about \LaTeX~packages}
Note that some of the environments present in the environment configuration
file require certain \LaTeX~packages to be available. Specifically, you should
use the following lines in the header:
\bdat
\usepackage{relsize,fancyvrb,moreverb,epsfig,framed}
\usepackage{color,listings,pythonhighlight}
\edat
The Ptex2tex package comes with a \LaTeX~style file
\code{ptex2tex.sty} containing the above commands, plus some tweaks of
packages.
If the \code{ptex2tex.sty} file is installed correctly, 
so that \LaTeX~can find it, one can simply use
\bdat
\usepackage{ptex2tex}
\edat
One can also simply
copy the style file to the current working directory. This file, as well as the
\code{.eps} files used for the Ptex2tex environments 'Warnings',
'Rules', and 'Summation', are located in the folder \code{latex} in the source
code for Ptex2tex. The installation script for the package will try to
install these to the computers' \LaTeX~folders, see the README file.

\subsection{Additional keywords}
In addition to the keywords starting with \code{@@@} and the keywords
associated with beginning
and ending Ptex2tex environments, there is another keyword available,
namely \code{\}\code{code}. This keyword can be used anywhere in the text, as
long as it is contained within a single line. This allows the use of any
character, we can for instance write 
\code{\}\code{code}\code{{}\code{@#_%}\verb!}! without \LaTeX~giving
us any errors. The result is similar to the \code{\}\code{texttt}
\LaTeX~command, but it has nicer typesetting of for instances underscores.
It is really just a slightly modified normal verbatim, and it
is based on the \code{\}\code{verb!}\code{...!} \LaTeX~command. The
modification consists of removing the escape character in \LaTeX, which is
\code{\}, in front of the characters \code{'\#'}, \code{'\%'}, \code{'\@'},
\code{'\$'} and \code{'\_'}. Writing \code{\}\code{code{}\code{\%}\smaller\verb!}!\larger~results
in the text for the rest of the line after the character \code{\%} being marked as a comment in most
editors, which is confusing. Writing
\code{\}\code{code{}\code{\$}\smaller\verb!}!\larger~results in the text for the rest
of the document being marked as a math environment. Using
\code{\}\code{code{}\code{\}\code{\%}\smaller\verb!}!\larger~and
\code{\}\code{code{}\code{\}\code{\$}\smaller\verb!}!\larger~instead gives us the same result
in the final document without this problem appearing in our editor.

\subsection{Mapping of keywords to environments}
The default mapping between Ptex2tex keywords and environments is included for
reference:
\bdat
# computer code in quote environment (gives a left margin):
ccq  = CodeIndented
# computer code with no left margin:
cc   = Code
# computer code with line numbering:
ccl  = CodeLineNo
# program box:
pro  = BlueBar
# computer code box (snippet, not complete program):
cod  = Blue
# computer code box (snippet, not complete program):
sni  = Blue
# data file:
dat  = CodeIndented
# data file snippet:
dsni = CodeIndented
# system commands (in terminal window):
sys  = CodeTerminal
# one-line system command (in terminal window):
slin = Code
# IPython interactive session:
ipy  = Code
# standard interactive python session:
py   = Code
# execution of a Python program ("run python"):
rpy  = CodeTerminal
# one-line program code:
plin = Code
# verbatim environment:
ver  = Verb
# warning box:
warn = Warnings
# tip box:
rule = Tip
# note box:
summ = Note
\edat

\subsection{Demo of the different environments}

There is a test script \code{testconfig.py} that can read a
configuration file and make a \LaTeX{} demo of all environments in
that file. Just run the script in a directory with a \code{.ptex2tex.cfg}
file. The result is a file \code{tmp_names} with definition of
new environment keys pointing to all environments found in the
configuration file. This \code{tmp_names} can be appended to your
\code{.ptex2tex.cfg}. The script \code{testconfig.py} also makes a
file \code{tmp_latex} which can be copied into any \LaTeX{} in
Ptex2tex format (i.e., a \code{.p.tex} file) to see a demo of
the environments. Below is such a \LaTeX{} demo.

% demo created by inserting the tmp_latex file generated by ../bin/testconfig.py
% first copy the latest .ptex2tex.cfg file:
%   cp ../lib/ptex2tex/ptex2tex.cfg .ptex2tex.cfg
% run 
%   python../bin/testconfig.py
% insert tmp_names at the end of .ptex2tex.cfg
% insert tmp_latex below


\noindent
Here is a demo of the environment 'Verb':
\bn1
v0 = 5                    # velocity
g = 9.81                  # acceleration of gravity
t = 0.6                   # time
y = v0*t - 0.5*g*t**2     # vertical position
print y
\en1


\noindent
Here is a demo of the environment 'CodeRule':
\bn2
v0 = 5                    # velocity
g = 9.81                  # acceleration of gravity
t = 0.6                   # time
y = v0*t - 0.5*g*t**2     # vertical position
print y
\en2


\noindent
Here is a demo of the environment 'CodeTerminal':
\bn3
v0 = 5                    # velocity
g = 9.81                  # acceleration of gravity
t = 0.6                   # time
y = v0*t - 0.5*g*t**2     # vertical position
print y
\en3


\noindent
Here is a demo of the environment 'Code':
\bn4
v0 = 5                    # velocity
g = 9.81                  # acceleration of gravity
t = 0.6                   # time
y = v0*t - 0.5*g*t**2     # vertical position
print y
\en4


\noindent
Here is a demo of the environment 'CodeLineNo':
\bn5
v0 = 5                    # velocity
g = 9.81                  # acceleration of gravity
t = 0.6                   # time
y = v0*t - 0.5*g*t**2     # vertical position
print y
\en5


\noindent
Here is a demo of the environment 'CodeIndented':
\bn6
v0 = 5                    # velocity
g = 9.81                  # acceleration of gravity
t = 0.6                   # time
y = v0*t - 0.5*g*t**2     # vertical position
print y
\en6


\noindent
Here is a demo of the environment 'CodeIndented10':
\bn7
v0 = 5                    # velocity
g = 9.81                  # acceleration of gravity
t = 0.6                   # time
y = v0*t - 0.5*g*t**2     # vertical position
print y
\en7


\noindent
Here is a demo of the environment 'PyHighlight':
\bn8
v0 = 5                    # velocity
g = 9.81                  # acceleration of gravity
t = 0.6                   # time
y = v0*t - 0.5*g*t**2     # vertical position
print y
\en8


\noindent
Here is a demo of the environment 'CodeGrayWhite1':
\bn9
v0 = 5                    # velocity
g = 9.81                  # acceleration of gravity
t = 0.6                   # time
y = v0*t - 0.5*g*t**2     # vertical position
print y
\en9


\noindent
Here is a demo of the environment 'CodeGrayWhite2':
\bn10
v0 = 5                    # velocity
g = 9.81                  # acceleration of gravity
t = 0.6                   # time
y = v0*t - 0.5*g*t**2     # vertical position
print y
\en10


\noindent
Here is a demo of the environment 'Warnings':
\bn11
Some message can be written here as ordinary
text.
\en11


\noindent
Here is a demo of the environment 'Tip':
\bn12
Some message can be written here as ordinary
text.
\en12


\noindent
Here is a demo of the environment 'Note':
\bn13
Some message can be written here as ordinary
text.
\en13


\noindent
Here is a demo of the environment 'Blue':
\bn14
v0 = 5                    # velocity
g = 9.81                  # acceleration of gravity
t = 0.6                   # time
y = v0*t - 0.5*g*t**2     # vertical position
print y
\en14


\noindent
Here is a demo of the environment 'BlueFrame':
\bn15
v0 = 5                    # velocity
g = 9.81                  # acceleration of gravity
t = 0.6                   # time
y = v0*t - 0.5*g*t**2     # vertical position
print y
\en15


\noindent
Here is a demo of the environment 'BlueBar':
\bn16
v0 = 5                    # velocity
g = 9.81                  # acceleration of gravity
t = 0.6                   # time
y = v0*t - 0.5*g*t**2     # vertical position
print y
\en16


\noindent
Here is a demo of the environment 'Red':
\bn17
v0 = 5                    # velocity
g = 9.81                  # acceleration of gravity
t = 0.6                   # time
y = v0*t - 0.5*g*t**2     # vertical position
print y
\en17


\noindent
Here is a demo of the environment 'RedFrame':
\bn18
v0 = 5                    # velocity
g = 9.81                  # acceleration of gravity
t = 0.6                   # time
y = v0*t - 0.5*g*t**2     # vertical position
print y
\en18


\noindent
Here is a demo of the environment 'RedBar':
\bn19
v0 = 5                    # velocity
g = 9.81                  # acceleration of gravity
t = 0.6                   # time
y = v0*t - 0.5*g*t**2     # vertical position
print y
\en19


\noindent
Here is a demo of the environment 'Yellow':
\bn20
v0 = 5                    # velocity
g = 9.81                  # acceleration of gravity
t = 0.6                   # time
y = v0*t - 0.5*g*t**2     # vertical position
print y
\en20


\noindent
Here is a demo of the environment 'Gray':
\bn21
v0 = 5                    # velocity
g = 9.81                  # acceleration of gravity
t = 0.6                   # time
y = v0*t - 0.5*g*t**2     # vertical position
print y
\en21


\noindent
Here is a demo of the environment 'Blue\_snippet':
\bn22
v0 = 5                    # velocity
g = 9.81                  # acceleration of gravity
t = 0.6                   # time
y = v0*t - 0.5*g*t**2     # vertical position
print y
\en22


\noindent
Here is a demo of the environment 'Blue\_singlepage':
\bn23
v0 = 5                    # velocity
g = 9.81                  # acceleration of gravity
t = 0.6                   # time
y = v0*t - 0.5*g*t**2     # vertical position
print y
\en23


\noindent
Here is a demo of the environment 'Gray\_singlepage':
\bn24
v0 = 5                    # velocity
g = 9.81                  # acceleration of gravity
t = 0.6                   # time
y = v0*t - 0.5*g*t**2     # vertical position
print y
\en24


\noindent
Here is a demo of the environment 'TiagoPy':
\bn25
v0 = 5                    # velocity
g = 9.81                  # acceleration of gravity
t = 0.6                   # time
y = v0*t - 0.5*g*t**2     # vertical position
print y
\en25


\noindent
Here is a demo of the environment 'TiagoCpp':
\bn26
v0 = 5                    # velocity
g = 9.81                  # acceleration of gravity
t = 0.6                   # time
y = v0*t - 0.5*g*t**2     # vertical position
print y
\en26


\noindent
Here is a demo of the environment 'FEniCS':
\bn27
v0 = 5                    # velocity
g = 9.81                  # acceleration of gravity
t = 0.6                   # time
y = v0*t - 0.5*g*t**2     # vertical position
print y
\en27


\noindent
Here is a demo of the environment 'FEniCSsmall':
\bn28
v0 = 5                    # velocity
g = 9.81                  # acceleration of gravity
t = 0.6                   # time
y = v0*t - 0.5*g*t**2     # vertical position
print y
\en28


\section{Support}
Please contact \code{ilmarw@simula.no} for bug reports, feature requests and
general help.
\end{document}
